\documentclass[11pt]{amsart}
\usepackage{geometry}                % See geometry.pdf to learn the layout options. There are lots.
\geometry{a4paper}                   % ... or a4paper or a5paper or ... 
%\geometry{landscape}                % Activate for for rotated page geometry
%\usepackage[parfill]{parskip}    % Activate to begin paragraphs with an empty line rather than an indent
\usepackage{graphicx}
\usepackage{amssymb}
\usepackage{epstopdf}
\DeclareGraphicsRule{.tif}{png}{.png}{`convert #1 `dirname #1`/`basename #1 .tif`.png}

\usepackage{listings}
\usepackage{xcolor}
\usepackage{hyperref}

\definecolor{dkgreen}{rgb}{0,0.6,0}
\definecolor{gray}{rgb}{0.5,0.5,0.5}
\definecolor{mauve}{rgb}{0.58,0,0.82}
%frame=tb
\lstset{frame=none ,
  language=Java,
  aboveskip=3mm,
  belowskip=3mm,
  showstringspaces=false,
  columns=flexible,
  basicstyle={\small\ttfamily},
  numbers=none,
  numberstyle=\tiny\color{gray},
  keywordstyle=\color{blue},
  commentstyle=\color{dkgreen},
  stringstyle=\color{mauve},
  breaklines=true,
  breakatwhitespace=true,
  tabsize=3
}

\title{Version-Control-System (VCS)}
\author{Grupp 2}
%\date{}                                           % Activate to display a given date or no date


%Changes

%--------------------------------------------------------------------------------------------------------
%DOCUMENT START
%--------------------------------------------------------------------------------------------------------
\begin{document}
\maketitle
\lstset{language=Java}

\section{Test}
\begin{lstlisting}
		public static void method(int code){
		
		//comments

		}
\end{lstlisting}

\section*{\"{O}versikt av v\r{a}r VCS}

\begin{enumerate}

\item \textbf{Git:} VCS:n som vi kommer anv\"{a}nda f\"{o}r konfigurationshantering av alla dokument (dvs. specifikationer , plan , kod , osv... ) som vi kommer beh\"{o}va skriva tillsammans. \\

\item \textbf{Github:} En gratis online server som anv\"{a}nder \textit{Git} d\"{a}r alla dokumenten och alla \"{a}ndringar av dokumenten kan lagras s\r{a} att de \"{a}r tillg\"{a}ngliga till alla i gruppen. \\

\item \textbf{LaTeX:} Ett programmeringsspr\r{a}k som beskriver inneh\r{a}llet av ett pdf-dokument. Vi kommer anv\"{a}nda LaTeX till att skriva v\r{a}ra textdokument f\"{o}r att allt vi skriver m\r{a}ste versionhanteras. Eftersom LaTeX \"{a}r ett programmeringsspr\r{a}k, kan LaTeX koden versionhanteras med \textit{Git} p\r{a} samma s\"{a}tt som javakoden.\\

\item Extra mjukvara f\"{o}r \textbf{Eclipse:}  
	\begin{itemize}
	\item \textbf{EGit:} Synkronisera Eclipse projekt med \textit{Github}
	\item \textbf{TeXlipse:} Kompilatorn f\"{o}r \textit{LaTeX} i Eclipse. 
	\end{itemize}
	
\end{enumerate}
$$ $$


\section{Installations- och inst\"{a}llningshandledning}

\subsection{Git/Github}
\begin{enumerate}
\item Skapa ett Github konto. \\
\end{enumerate}

\subsection{LaTeX}
\begin{enumerate}
\item[] {\color{red}{Viktig: Det kan ta n\r{a}gra timmar att ladda ned, installera, synkronisera och uppdatera allting.}}
\item Installera en LaTeX distribution. Vi rekommenderar de h\"{a}r, men du f\r{a}r v\"{a}lja vilken distribution som helst.
	\begin{itemize}
	\item Windows: proTeXt {\color{blue}{\underline{\url{https://www.tug.org/protext/}}}}
	\item Mac: MacTeX {\color{blue}{\underline{\url{https://www.tug.org/mactex/}}}}
	\item Unix/GNU/Linux: TeX Live {\color{blue}{\underline{\url{https://www.tug.org/texlive/}}}} \\
	\end{itemize}
\item L\"{a}gg m\"{a}rke till var LaTeX installeras p\r{a} datorn eftersom Eclipse kommer beh\"{o}va veta. S\"{a}rskilt bin mappen.
\end{enumerate}

\subsection{TeXlipse}
\begin{enumerate}

\item[] Mer detaljerat: {\color{blue}{\underline{\href{http://texlipse.sourceforge.net/manual/installation.html}{texlipse.sourceforge.net}}}} \\

\item \"{O}ppna Eclipse  

\item Preferences $\rightarrow$ Install/Update $\rightarrow$ Available Software Sites
	\begin{itemize}
	\item Tryck ``Add'' knappen och l\"{a}gga till $<$ http://texlipse.sourceforge.net $>$ i ``Available Software Sites'' listan. 
	\item Name: TeXlipse 
	\item URL: http://texlipse.sourceforge.net
	\end{itemize}

\item Help $\rightarrow$ Install New Software
	\begin{itemize}
	\item Hitta TeXlipse i ``work with'' listan och installera allting.
	\end{itemize}
	
\item Preferences $\rightarrow$ TeXlipse $\rightarrow$ Builder Settings
	\begin{itemize}
	\item St\"{a}lla in ``Bin directory''. Dvs. Ange s\"{o}kv\"{a}g till bin mappen av din LaTeX distribituion. 
	\item F\"{o}nstret ovan skulle \"{a}ndra sig med att l\"{a}gga till s\"{o}kv\"{a}gar till de flesta av kompilatorna, men det viktigaste \"{a}r att kontrollera att det finns s\"{o}kv\"{a}gar till LaTeX och PdfLatex. 
	\end{itemize}
	
\item Preferences $\rightarrow$ TeXlipse $\rightarrow$ Viewer Settings
	\begin{itemize}
	\item Kontrollera att det redan finns en ``default viewer''.
	\item Standarden skulle vara den som kom med din LaTeX distribution.
	\item Om det inte finns eller du vill anv\"{a}nda ett annat program, \"{a}ndra det h\"{a}r.
	\end{itemize}

\end{enumerate}

\subsection{EGit}
\begin{enumerate}
\item[] Didn't finish the whole instruction thing today, but here's a good place to start:  
{\color{blue}{\underline{\url{https://wiki.eclipse.org/EGit/User_Guide}}}}

\item Preferences $\rightarrow$ Install/Update $\rightarrow$ Available Software Sites
	\begin{itemize}
	\item Tryck ``Add'' knappen och l\"{a}gga till $<$ http://download.eclipse.org/egit/updates $>$ i ``Available Software Sites'' listan. 
	\item Name: EGit 
	\item URL: http://download.eclipse.org/egit/updates
	\end{itemize}

\item Help $\rightarrow$ Install New Software
	\begin{itemize}
	\item Hitta EGit i ``work with'' listan och installera allting som kan
	instaleras.
	\end{itemize}
	
%------ | ------
%------ v ------
%CHANGES
%CHANGES
%------ ^ ------
%------ | ------

\end{enumerate}






\end{document}

